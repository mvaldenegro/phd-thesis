\chapter{Conclusions and Future Work}
\label{chapter:conclusions}
Marine debris is ubiquitous in the world's oceans, rivers, and lakes, caused by the massive production of disposable products, mostly made out of plastic and glass. Marine debris has been found up to 4000 meters deep in the Mariana Trench.  There is evidence that surface and submerged marine debris \cite{iniguez2016marine} are harmful to marine environments, and efforts to reduce the amount of debris getting into the sea are underway \cite[1em]{mcilgorm2008understanding}. This thesis proposes that Autonomous Underwater Vehicles can be used to survey and detect submerged marine debris with the long term goal of recover and cleaning up.

In this thesis we have developed techniques to detect and recognize marine debris in Forward-Looking sonar images. But more importantly, we have proposed the use of Autonomous Underwater Vehicles to survey and detect submerged marine debris with sonar sensor. We show that detecting marine debris is harder than other objects such as mine like objects (MLOs), because techniques commonly used for MLOs fail to generalize well for debris.

We did this through several research lines, and encouraged by recent advances in Deep Neural Networks (DNNs), we adopt this framework and develop our own neural network models for Forward-Looking Sonar images.

We captured a small dataset of 2069 Forward-Looking Sonar images using a ARIS Explorer 3000 sensor, containing 2364 labeled object instances with bounding box and class information. This dataset was captured from a selected set of household and marine objects, representing non-exhaustive instances of marine debris. The data capture was realized from a water tank, as we had issues with the sensor in a real world underwater environment. We use this dataset for training and evaluation of all machine learning models in this thesis.

The first research line considers Forward-Looking Sonar image classification, as a proxy for posterior performance in object detection. A state of the art template matching classifier obtains $98.1 \%$ accuracy using sum of squared differences similarity, and $93.0 \%$ for cross-correlation similarity, which pales in comparison with a DNN classifier that obtains up to $99.7 \%$ accuracy. Both template matching classifiers require 150 templates per class in order to obtain this result, which implies that the whole training set is memorized. A DNN requires far fewer trainable parameters (up to 4000) to achieve a better accuracy on our dataset.

We have also developed neural network architectures for sonar image classification that generalize well and are able to run in real-time in low power platforms, such as the Raspberry Pi 2. This is done by carefully designing a neural network, based on the Fire module \cite[-7em]{iandola2016squeezenet}, that has a low number of parameters but still having good classification performance, requiring less computational resources. This shows that a DNN can perform well even in the constrained computing environments of an AUV.

As a second research line, we evaluated the practical capabilities of in Forward-Looking sonar image classification DNNs with respect to the amount of training data required for generalization, the input image size, and the effect of transfer learning. We find that when scaling down the training set, classification performance drops considerably, but when using transfer learning to first learn features and use them with an SVM classifier, classification performance increases radically. Even with a training set of one sample per class, it is possible to generalize close to $90 \%$ accuracy when learning features from different objects.
With respect of input image size, we find that classifier architectures with fully connected layers are able to generalize well even if trained with a dataset of $16 \times 16$ images, but this is only possible if using the Adam optimizer and Batch Normalization. Using SGD and Dropout has the effect of producing a linear relation between accuracy and image size.

Our architectures based on the Fire module do not have the same behavior, and suffer from decreased performance with small image sizes, and with smaller datasets. We believe additional research is needed to understand why this happens, as a model with less parameters should be easier to train with less data.

The third research line is the problem of matching two in Forward-Looking Sonar image patches of the same size. This is a difficult problem as multiple viewpoints will produce radically different sonar images. We frame this problem as binary classification. Using two-channel neural networks we obtain up to $0.91$ Area under the ROC Curve (AUC), which is quite high when compared with state of the art keypoint detection techniques such as SIFT, SURF, and AKAZE, whom obtain $0.65-0.80$ AUC. This result holds when a DNN is trained with one set of objects, and tested with a different one, where we obtain $0.89$ AUC.

As fourth research line we explored the use of detection proposal algorithms in order to detect objects in in Forward-Looking sonar images, even when the object shape is unknown or there are no training samples of the object. Our methods based on DNNs can achieve up to $95 \%$ recall, while generalizing well to unseen objects. We even used data captured by the University of Girona, using an older version of the same sonar sensor, and noticed that detections are appropriate for the chain object. We also evaluated the number of output detections/proposals that are required for good generalization, and our methods obtain high recall with less than 100 proposals per image, while a baseline template matching that we built requires at least 300 proposals per image, while only achieving $90 \%$ recall. In comparison, state of the art proposals techniques like EdgeBoxes and Selective Search require thousands of detections to reach comparable recall levels.

As the final technical chapter of this thesis, we showcase two applications of our methods in Forward-Looking Sonar images. The first is performing object detection by classifying proposals, in an end-to-end DNN architecture that learns both to detect proposals through objectness, and classify them with an additional output head. This method obtains $80 \%$ correct classifications, which can be improved to $90 \%$ accuracy by replacing the classifier with an SVM trained on the learned features.

The second application is object tracking in Forward-Looking Sonar images. We build a simple tracker by combining our detection proposals algorithm with the matching system, from where an object is first detected and continually matched to the first detection across time. This method outperforms a cross-correlation based matcher that we used as baseline in the correctly tracked frames metric..

Overall we believe that our results show that using Deep Neural Networks is promising for the task of Marine Debris detection in Forward-Looking Sonar images, and that they can be successfully used in Autonomous Underwater Vehicles, even when computational power is constrained.

We expect that interest on neural networks will increase in the AUV community, and they will be used for detection of other kinds of objects in marine environments, specially as no feature engineering or object shadow/highlight shapes are needed. For example, our detection proposals algorithm has the potential of being used to find \textit{anomalies} in the seafloor, which can be useful to find wrecked ships and airplanes at sea.

\section{Future Work}

There is plenty of work that can be done in the future to extend this thesis.

The dataset that we captured does not fully cover the large variety of marine debris, and only has one environment: the OSL water tank. We believe that a larger scientific effort should be made to capture a ImageNet-scale dataset of marine debris in a variety of real-world environments. This requires the effort of more than just one PhD student. A New dataset should consider a larger set of debris objects, and include many kinds of \textit{distractor} objects such as rocks, marine flora and fauna, with a richer variety of environments, like sand, rocks, mud, etc.

Bounding Box prediction seems to be a complicated issue, as our informal experiments showed that with the data that is available to us, it does not converge into an usable solution. Due to this we used fixed scale bounding boxes for detection proposals, which seem to work well for our objects, but it would not work with larger scale variations. A larger and more varied dataset could make a state of the art object detection method such as SSD or Faster R-CNN work well, from where more advanced detection proposal methods could be built upon.

We only explored the use of the ARIS Explorer 3000 Forward-Looking Sonar to detect marine debris, but other sensors could also be useful. Particularly a civilian Synthetic Aperture Sonar could be used for large scale surveying of the seafloor, in particular to locate regions where debris accumulates, and AUVs can target these regions more thoroughly.
Underwater laser scanners could also prove useful to recognize debris or to perform manipulation and grasping for collection, but these sensors would require new neural network architectures to deal with the highly unstructured outputs that they produce. There are newer advances in neural networks\cite{qi2017pointnet} that can process point clouds produced by laser sensors, but they are computationally expensive.

Another promising approach to detect marine debris is that instead of using object detection methods, which detect based visual appearance on an image, a sensor or algorithm could recognize the materials that compose the object under inspection, from where debris could be identified if the most common material is plastic or metal.

This technique could cover a more broad set of objects and could be generally more useful for other tasks, like finding ship or plane wrecks. This approach would require a radically new sonar sensor design that can use multiple frequencies to insonify an object.  In collaboration with Mariia Dmitrieva we have some research\cite[-5em]{dmitrieva2017object} in this direction.

Last but not least, we did not cover the topic of object manipulation in this thesis. Once objects are detected, it is a must to capture them using a manipulator arm with an appropriate end effector. This would require further research in designing an appropriate end effector that can capture any kind of marine debris, and then perception algorithms to robustly do pose estimation, and finally perform the grasping motion to capture the object.

The biggest research issue in this line is that manipulation has to be performed without making any assumptions in object shape or grasping points. All the grasping information has to be estimated by a robust perception algorithm directly from the perceived object, in order for it to be available at runtime. Any kind of assumption made on the object's structure will limit the kind of debris that can be grasped.